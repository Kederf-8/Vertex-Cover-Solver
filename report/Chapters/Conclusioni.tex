\section{Conclusioni}
\label{sec:Conclusioni}


Nel corso di questo progetto è stato sviluppato un solver per il problema del \textit{Weighted Vertex Cover} basato su un’implementazione della \textit{Tabu Search}. L’obiettivo principale era applicare una metaeuristica efficace per trovare soluzioni di alta qualità in tempi computazionali ragionevoli, anche su istanze di dimensioni medio-grandi. 

\subsection{Riepilogo del Lavoro Svolto}

L'attività svolta si è articolata nelle seguenti fasi principali:

\begin{itemize}
    \item Definizione della soluzione come insieme di nodi che coprono tutti gli archi del grafo, utilizzando una rappresentazione compatta ed efficiente.
    \item Implementazione di un algoritmo \textit{Tabu Search}, con i seguenti meccanismi:
    \begin{itemize}
        \item \textbf{Short-term memory}: per evitare il ritorno immediato a soluzioni già visitate, migliorando la capacità di esplorazione.
        \item \textbf{Aspiration criterion}: che consente di accettare mosse tabu se la soluzione migliora il miglior risultato globale.
    \end{itemize}
    \item Analisi dei parametri chiave dell'algoritmo (\textit{tabuTenure}, \textit{maxIterations}, \textit{maxNoImprovement}) e valutazione del loro impatto sui risultati ottenuti.
    \item Sperimentazione su istanze di diverse dimensioni, con valutazione della qualità delle soluzioni e dei tempi di esecuzione.
\end{itemize}

\subsection{Scelte Vincenti e Punti di Forza}

L'implementazione sviluppata presenta alcune differenze significative rispetto alla \textit{Tabu Search} classica, con diverse scelte progettuali che si sono rivelate vincenti:

\begin{itemize}
    \item \textbf{Generazione controllata del vicinato}: ogni mossa consiste nell'aggiunta o rimozione di un solo nodo, mantenendo la soluzione sempre ammissibile. Ciò garantisce stabilità nella ricerca, evitando la necessità di correzioni successive.
    \item \textbf{Gestione efficiente della lista tabu}: ogni mossa è memorizzata come una semplice chiave testuale (\texttt{"ADD\_x"} o \texttt{"REMOVE\_x"}), riducendo il consumo di memoria e la complessità computazionale.
    \item \textbf{Aspiration criterion ottimizzato}: se una mossa è tabu ma migliora la migliore soluzione globale, viene comunque accettata, permettendo all'algoritmo di sfuggire ai minimi locali più evidenti.
    \item \textbf{Selezione del vicino robusta}: in assenza di candidati validi, viene scelto un vicino casuale, evitando stalli nella ricerca e garantendo una migliore esplorazione dello spazio delle soluzioni.
    \item \textbf{Criterio di arresto bilanciato}: l’algoritmo si interrompe quando non si ottengono miglioramenti per un numero definito di iterazioni (\textit{maxNoImprovement}), prevenendo esecuzioni eccessivamente lunghe senza progresso.
\end{itemize}

I risultati sperimentali confermano che queste strategie permettono di ottenere soluzioni di qualità con tempi computazionali contenuti, rendendo l'algoritmo scalabile anche per istanze di grafi con migliaia di archi.

\subsection{Sviluppi Futuri}

Tra le possibili estensioni e miglioramenti dell'algoritmo si evidenziano:

\begin{itemize}
    \item \textbf{Integrazione di una Long-term Memory}: per raccogliere informazioni sulle frequenze di selezione dei nodi e favorire una maggiore diversificazione della ricerca.
    \item \textbf{Ibridazione con altre metaeuristiche}: combinando la \textit{Tabu Search} con algoritmi genetici o strategie di \textit{simulated annealing} per migliorare l’esplorazione dello spazio delle soluzioni.
    \item \textbf{Parallelizzazione dell'algoritmo}: l'esecuzione parallela della generazione del vicinato e della valutazione delle soluzioni potrebbe accelerare significativamente i tempi di calcolo, sfruttando CPU multi-core o GPU.
    \item \textbf{Test su istanze di grandi dimensioni}: per valutare strategie di pruning o di riduzione del vicinato in grafi con decine di migliaia di nodi.
\end{itemize}

\subsection{Considerazioni Finali}

L'esperienza acquisita in questo progetto dimostra come la \textit{Tabu Search} sia un potente strumento per risolvere problemi NP-hard come il \textit{Weighted Vertex Cover}. Pur non garantendo l'ottimalità delle soluzioni, l'implementazione proposta si è dimostrata competitiva rispetto ad altre tecniche euristiche più semplici, producendo risultati affidabili e scalabili.
Le scelte progettuali adottate hanno permesso di bilanciare esplorazione e intensificazione della ricerca, migliorando la qualità delle soluzioni ottenute rispetto a un'implementazione standard della \textit{Tabu Search}. Questo lavoro rappresenta quindi una base solida per futuri miglioramenti e per l’applicazione della metaeuristica a problemi più complessi.

% Fine capitolo Conclusioni
