\section{Algoritmo}
\label{sec:Algoritmo}

\subsection{Panoramica Generale}

L'algoritmo adottato è una \textit{Tabu Search}, una metaeuristica che evita cicli memorizzando le mosse recenti in una \textit{tabu list}. Come descritto nell'Introduzione, il problema viene affrontato tramite una ricerca locale iterativa, partendo da una soluzione iniziale e modificandola per minimizzare il costo.

Ad ogni iterazione, vengono generate soluzioni vicine valutate tramite una funzione costo. La \textit{tabu list} impedisce di ripetere modifiche recenti per un numero prefissato di iterazioni, bilanciando esplorazione e ottimizzazione. Nei prossimi paragrafi verranno descritti i dettagli implementativi dell’algoritmo.

\subsection{Struttura Della Tabu Search}
\label{sec:TabuSearch}

L'algoritmo si basa su una ricerca locale iterativa, mantenendo una soluzione corrente e aggiornandola esplorando il vicinato. La qualità di ogni soluzione è valutata tramite una funzione costo, mentre una \textit{tabu list} impedisce il ripristino di mosse recenti, favorendo l'esplorazione di nuove configurazioni.

Gli elementi principali dell'algoritmo includono:
\begin{itemize}
    \item \textbf{Soluzione corrente}: insieme di vertici selezionati nel \textit{vertex cover}.
    \item \textbf{Migliore soluzione}: la configurazione con costo minimo trovata finora.
    \item \textbf{Tabu list}: insieme di mosse vietate per un numero limitato di iterazioni.
    \item \textbf{Vicinato}: soluzioni ottenute aggiungendo o rimuovendo vertici.
\end{itemize}

L'algoritmo prosegue fino al raggiungimento di un criterio di arresto, basato su un numero massimo di iterazioni o sull'assenza di miglioramenti per un intervallo predefinito. Il funzionamento dettagliato, inclusa la gestione della \textit{tabu list} e dei parametri principali (\texttt{tabuTenure}, \texttt{maxIterations}, \texttt{maxNoImprovement}), è illustrato nello pseudocodice della sezione successiva.

Tra i parametri fondamentali troviamo:
\begin{itemize}
    \item \textbf{\emph{maxIterations}:} numero massimo di iterazioni complessive.
    \item \textbf{\emph{tabuTenure}:} numero di iterazioni per cui una mossa rimane nella tabu list.
    \item \textbf{\emph{maxNoImprovement}:} numero massimo di iterazioni consecutive senza miglioramenti, oltre cui fermare l'algoritmo.
\end{itemize}

\subsection{Pseudocodice}

L'algoritmo segue un ciclo iterativo in cui si generano soluzioni vicine e si applicano criteri di selezione basati sulla \textit{tabu list} e sulla qualità della soluzione. La funzione di valutazione guida la ricerca, mentre i criteri di aspirazione permettono di ignorare vincoli tabu se si trova una soluzione migliore.

Il funzionamento è riassunto nel seguente pseudocodice:

\begin{lstlisting}[language=Python, caption=Pseudocodice della Tabu Search]
FUNCTION TABU_SEARCH(Graph, maxIter, tabuTenure, maxNoImprovement):
    bestSolution = INITIAL_SOLUTION(Graph)
    currentSolution = bestSolution.copy()
    tabuList = emptySet
    iterations = 0
    noImprovement = 0

    WHILE iterations < maxIter AND noImprovement < maxNoImprovement:
        neighbors = GENERATE_NEIGHBORS(currentSolution)
        bestNeighbor = SELECT_BEST(neighbors, tabuList, bestSolution)

        IF bestNeighbor is NULL:
            BREAK
        
        UPDATE_TABU_LIST(tabuList)
        tabuList.add(DETERMINE_MOVE(currentSolution, bestNeighbor), tabuTenure)
        
        currentSolution = bestNeighbor.copy()

        IF cost(currentSolution) < cost(bestSolution):
            bestSolution = currentSolution.copy()
            noImprovement = 0
        ELSE:
            noImprovement += 1

        iterations += 1

    RETURN bestSolution
\end{lstlisting}

Le principali operazioni sono:
\begin{itemize}
    \item \texttt{INITIAL\_SOLUTION}: genera una configurazione iniziale ammissibile.
    \item \texttt{GENERATE\_NEIGHBORS}: produce soluzioni vicine.
    \item \texttt{SELECT\_BEST}: sceglie la soluzione migliore rispettando le regole tabu.
    \item \texttt{UPDATE\_TABU\_LIST}: aggiorna la memoria tabu rimuovendo mosse scadute.
    \item \texttt{DETERMINE\_MOVE}: identifica la modifica necessaria per la transizione tra soluzioni.
\end{itemize}

L'algoritmo si arresta se il numero massimo di iterazioni è raggiunto o se non si verificano miglioramenti per un certo numero di iterazioni consecutive.

\subsection{Esecuzione e Parametri Principali}

In fase di esecuzione, l'algoritmo arresta la ricerca in presenza di due condizioni:
\begin{enumerate}
    \item Il \emph{numero totale di iterazioni} raggiunge \emph{maxIterations}.
    \item Non si osservano miglioramenti alla \emph{bestSolution} per più di \emph{maxNoImprovement} iterazioni consecutive.
\end{enumerate}

Le prestazioni (sia in termini di tempo di esecuzione che di qualità delle soluzioni) dipendono fortemente dalla scelta di parametri come \emph{maxIterations}, \emph{tabuTenure}, \emph{maxNoImprovement} e dalla definizione delle mosse nel \emph{vicinato}.

In questo lavoro, si è analizzato un range di valori per i parametri chiave (ad esempio, \emph{tabuTenure} variato tra 5 e 7, \emph{maxNoImprovement} tra 10 e 30), confrontando i risultati sia in termini di costi ottenuti sia di stabilità dell'algoritmo. I dettagli di tali sperimentazioni sono riportati nel Capitolo~\ref{sec:Risultati}.

% (Finisce il capitolo Algoritmo)
