\section{Introduzione}

Il \emph{Vertex Cover} è un problema classico della teoria dei grafi e dell’ottimizzazione combinatoria, in cui si cerca un insieme di vertici tale da \emph{coprire} (o intercettare) tutti gli archi di un grafo. Nella sua variante pesata (\emph{Weighted Vertex Cover}), ad ogni vertice è associato un costo (o peso) e l’obiettivo consiste nel minimizzare la somma dei pesi dei vertici selezionati, garantendo al contempo la copertura di tutti gli archi. Questo problema risulta \emph{NP-hard}, rendendolo particolarmente sfidante dal punto di vista computazionale, soprattutto per istanze di grandi dimensioni.

Nel presente progetto, si è scelto di affrontare il Weighted Vertex Cover mediante una \emph{metaeuristica di tipo Tabu Search}, al fine di ottenere soluzioni di buona qualità in tempi ragionevoli, anche su istanze con dimensioni non trascurabili. La Tabu Search, introdotta da Fred Glover, è un approccio iterativo che combina una ricerca locale sistematica con una struttura di memoria (detta \emph{tabu}) in grado di gestire sia la cosiddetta \emph{short-term memory} (con lo scopo di evitare cicli o ritorni su soluzioni già visitate), sia la \emph{long-term memory} (orientata ad aumentare la diversificazione e l’esplorazione dello spazio delle soluzioni).

Nel prosieguo, verranno illustrate le principali caratteristiche dell’algoritmo, le scelte progettuali e implementative adottate --- con particolare attenzione ai parametri chiave (\emph{maxIterations}, \emph{tabuTenure} e \emph{maxNoImprovement}) --- e i risultati sperimentali ottenuti. Verranno infine discussi possibili sviluppi futuri e riflessioni personali emerse dal lavoro svolto.
